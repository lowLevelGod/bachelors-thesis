\begin{abstractpage}

    \begin{abstract}{romanian}  
    
        Detecţia anomaliilor este un subiect ce apare frecvent în cercetarea din 
        domeniul învăţării automate. Acest fapt se datorează multitudinii de 
        aplicaţii într-o gamă largă de industrii. 
    
        Totuşi, găsirea unei tehnici bune de rezolvare a problemei este
        dificilă în cele mai multe cazuri, 
        întrucât este costisitor sau aproape imposibil să obţii date 
        despre evenimentele rare ce merită semnalate. 
    
        În această lucrare, explorăm un set de date ce conţine 
        informaţii despre tranzacţii bancare
        folosind 3 algoritmi de învăţare automată nesupervizată:
        "One Class SVM", "Gaussian Mixture Model" şi "Kernel Density Estimation". 
        Performanţa acestora din urmă 
        este comparată prin metricile: "accuracy", "precision", "recall" şi "f1 score".
    
        Obiectivul este detectarea tranzacţiilor frauduloase.    
    \end{abstract}
    
    \begin{abstract}{english}
        
        Anomaly detection is a topic appearing frequently in machine learning research.
        This is due to its many applications in various industrial settings.
    
        However, finding a good algorithm to solve this problem is difficult 
        in most cases, since obtaining data about rare events that are worth reporting
        is costly or almost impossible.
    
        In this paper, we analyze a dataset containing information about 
        bank transactions using 3 unsupervised machine learning 
        algorithms: "One Class SVM", "Gaussian Mixture Model" and 
        "Kernel Density Estimation". 
        Their performance is compared using the following metrics: 
        "accuracy", "precision", "recall" and "f1 score".
    
        Our objective is detecting fraudulent transactions.
    
    \end{abstract}
    
    \end{abstractpage}