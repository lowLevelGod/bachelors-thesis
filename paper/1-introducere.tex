\chapter{Introducere}

\section{Definiţia anomaliilor}

O \textbf{anomalie} este o entitate ce diferă semnificativ de restul enităţilor din 
setul de date. Definiţia lui Hawkins este urmatoarea\cite{hawkins1980identification}:
\textit{"O anomalie este o observaţie ce deviază atât de mult faţă de restul observaţiilor,
încât să creeze suspiciunea că a fost generată de un mecanism diferit"}.

Datele sunt colectate prin observarea unor procese, de preferat din viaţa reală, 
precum funcţionarea
unui motor sau traficul pe o reţea de calculatoare. În majoritatea timpului, procesele
generează date ce corespund unei derulări normale, dar în anumite cazuri acestea se 
comportă anormal şi astfel apar anomaliile. 

\textit{"Normalitatea"} observaţiilor este definită
de oameni. Algoritmii de învăţare automată pot să semnaleze potenţialele
abateri, dar în final, decizia rămâne a omului.

\section{Avantajele învăţării automate}

Când sarcinile ce trebuie îndeplinite periodic, necesare bunei funcţionări a unei 
companii spre exemplu, încep să ocupe o mare parte din timpul oamenilor, apare 
nevoia automatizării. 

În general, soluţiile ce includ stabilirea unui set de reguli
sunt de preferat, datorită simplităţii lor. Un exemplu banal ar fi verificarea 
siguranţei unei parole introduse de utilizator la crearea unui cont pe o platformă
web. Este suficient să verificăm dacă parola are un număr minim precizat de caractere 
sau dacă conţine anumite simboluri ce reduc vulnerabilitatea în faţa unui atac cibernetic.

Din păcate, multe din problemele cu care ne confruntăm nu au soluţii simple. 
Aici intervine învăţarea automată ce încearcă să aproximeze posibilul set de reguli, 
prea complex pentru a fi definit, ce ar putea rezolva problema. Am putea spune că 
această tehnică "învaţă" setul de reguli 
din datele puse la dispoziţie, de unde şi denumirea. Evident, o aproximare 
nu are cum să fie perfectă, dar este suficient de bună încât să scutească oamenii de 
o mare parte din treabă. În cazul tranzacţiilor bancare, ce probabil sunt efectuate 
în număr de sute de milioane pe zi, nu este greu de observat că verificarea manuală
de către oameni este imposibilă, iar găsirea unui set simplu de reguli care să 
le caracterizeze din nou este o problemă dificilă. Totuşi, învăţarea automată 
ne poate ajuta să minimizăm fraudele ce trec nedetectate de la milioane la câteva zeci
pe zi.

\section{Caracterizarea anomaliilor nu este viabilă}

Un exemplu ce ilustrează de ce obţinerea de informaţii
despre anomalii este costisitoare ar putea fi activitatea unui
motor defect.

În acest caz, avem 2 variante:
\begin{itemize}
    \item Una dintre ele 
        este să \textbf{simulăm} 
        într-un fel sau altul comportamentul unui motor defectuos, dar 
        se observă că această operaţiune este greu de realizat şi chiar dacă am reuşi 
        să o ducem la capăt, datele colectate nu ar fi autentice şi ar putea duce 
        la o reprezentare incorectă a anomaliilor. 

    \item A doua variantă implică \textbf{sabotarea 
        intenţionată} a motorului pentru a obţine datele, dar acest lucru produce un 
        cost prea mare de cele mai multe ori. 
\end{itemize}

Este evident că definirea anomaliei în acest context nu este viabilă, iar 
cazul nu este singular. Tranzacţiile frauduloase ce constituie tema acestei lucrări 
suferă de aceleaşi probleme, ba mai mult, nici măcar nu avem opţiunea de a 
sabota intenţionat derularea normală a tranzacţiilor. În schimb, 
este mult mai uşor şi cel mai probabil 
implică un cost aproape inexistent, să urmărim activitatea unui motor 
în stare bună de funcţionare şi să folosim datele respective pentru a 
defini ce înseamnă o observaţie normală.

\section{Aplicaţii ale detecţiei anomaliilor}

În aproape toate domeniile apare problema semnalării unor fenomene sau evenimente 
ieşite din comun ce necesită atenţia unei persoane pentru analiză. Mai jos prezentăm
doar câteva din cazurile unde detecţia anomaliilor este necesară, cât şi ce ar 
reprezenta o anomalie pentru fiecare:

\begin{itemize}
    \item \textbf{Fraude financiare}: ne dorim să semnalăm comportamentul ciudat 
    observat într-o serie de tranzacţii, precum activitatea generată de 
    o persoana care foloseşte cardul de credit al altei persoane în mod 
    neautorizat. Această aplicaţie constituie şi obiectul lucrării noastre.
    \item \textbf{Intruziune în reţele de calculatoare}: vrem să semnalăm activitatea
    neobişnuită ce poate indica un potenţial atac cibernetic sau accesul 
    neautorizat al unui terţ maliţios.
    \item \textbf{Controlul calităţii în manufactură}: suntem interesaţi să monitorizăm 
    procesul de producţie pentru a raporta eventualele defecţiuni ce ar afecta 
    calitatea produsului.
    \item \textbf{Domeniul medical}: vrem să identificăm anomalii în analizele de sânge, 
    semnele vitale sau imaginile medicale ale pacientului pentru a depista şi preveni
    anumite afecţiuni.
    \item \textbf{Reţele sociale}: ne dorim să raportăm automat conţinutul postat ce 
    poate include remarci jignitoare, precum comentariile rasiste sau xenofobe,
    astfel fiind de ajutor în moderarea materialului încărcat pe platforme. 
\end{itemize}

\section{Detecţia anomaliilor vs clasificare clasică}

\subsection{Greşeală comună}

La prima vedere, aceste două probleme par sa coincidă şi ne determină
să ne întrebăm de ce a mai apărut un nou domeniu de cercetare, anume detecţia
anomaliilor, când deja avem la dispoziţie atâtea rezultate utile pentru 
problema clasificării. Din păcate, dacă nu cunoaştem numărul de clase ce pot apărea 
sau nu avem un eşantion reprezentativ pentru toate tipurile de clase, fapt ce se 
întâmplă deseori în practică, nu putem utiliza un clasificator clasic. De exemplu, 
în cazul tranzacţiilor bancare, nu putem şti cu certitudine care sunt toate 
tipurile de fraudă ce pot apărea şi nici nu avem un număr suficient de exemple 
pentru a defini măcar tipurile de fraudă pe care le-am identificat până acum.
Prin urmare, este mult mai uşor să modelăm problema în jurul unei singure clase 
de referință, decât să încercăm definirea unei clase adiţionale ce de fapt 
este compusă dintr-o multitudine de subclase necunoscute.


\subsection{Scopul detecţiei anomaliilor}

Diferenţa constă în faptul că detecţia anomaliilor are la bază \textbf{o singură clasă
de referință}. Observaţia fie aparţine acestei clase, fie aparţine oricărei alte clase 
care este diferită de aceasta din urmă. Multitudinea de "clase diferite" ne indică
faptul că este greu să definim ce înseamnă \textit{"diferit de normal"}, dar este relativ uşor
să definim ce înseamnă \textit{"normal"}.

Un exemplu trivial ar putea fi să detectăm dacă animalul din imagine este un câine sau nu.
În acest caz, normal înseamnă câine, iar anormal înseamnă orice alt animal care nu este
câine. Se observă că este relativ uşor să caracterizăm conceptul de câine, în timp ce 
definirea conceptului de "diferit de câine" este complexă, fapt ce ar pune în dificultate
un clasificator clasic care prin definiţie are nevoie de un număr cunoscut de clase 
cu observaţiile reprezentative aferente.

\subsection{Scopul clasificării clasice}

La clasificarea clasică, clasele ce trebuie identificate sunt bine 
definite şi de cele mai multe ori, trasăturile lor se suprapun în mai multe locuri. De 
asemenea, această problemă acordă o importanţă \textbf{egală} tuturor categoriilor
implicate, pe când 
eşecul de a semnala o anomalie este în general mult mai dăunător faţă de 
detectarea unei observaţii normale ca fiind anormală. 

Pentru a continua exemplul precedent cu imaginile cu animale, putem reformula problema 
ca de această dată să diferenţiem între pisici şi câini. Deja se vede că ambele clase 
sunt mult mai bine definite şi putem găsi atât trăsături comune, precum nasul, cât şi 
trăsături definitorii, precum mustăţile. Vrem să detectăm la fel de bine ambele animale, 
pe când într-o problemă clasică de detecţie a anomaliilor, precum identificarea 
tranzacţiilor frauduloase, vrem preponderent să nu ratăm evenimentele cu 
caracter maliţios.