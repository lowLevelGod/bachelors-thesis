\chapter{Introducere}

\section{Definiţia anomaliilor}

O \textbf{anomalie} este o entitate ce difera semnificativ de restul enităţilor din 
setul de date. Definiţia lui Hawkins este urmatoarea\cite{Identification of Outliers}:
\textit{"O anomalie este o observaţie ce deviază atât de mult faţă de restul observaţiilor,
încât să creeze suspiciunea ca a fost generată de un mecanism diferit"}.

Datele sunt colectate prin observarea unor procese din viaţa reală, precum funcţionarea
unui motor sau traficul pe o reţea de calculatoare. În majoritatea timpului, procesele
generează date ce corespund unei derulări normale, dar în anumite cazuri acestea se 
comporta anormal si astfel apar anomaliile. Un exemplu ar putea fi activitatea unui
motor defect.

\textit{"Normalitatea"} observaţiilor este definită
de oameni. Algoritmii de învăţare automată pot să semnaleze potenţialele
abateri, dar în final, decizia rămâne a omului.

\section{Tipuri de detecţie a anomaliilor}

\subsection{Outlier detection}

Această abordare este utilă atunci cand avem un set de date \textbf{"poluat"} cu 
observaţii anormale şi ne dorim sa extragem din el doar porţiunea ce conţine 
observaţii normale.

\subsection{Novelty detection}

Această abordare este utilă atunci cand avem un set de date \textbf{"curat"} (fără 
anomalii) şi ne dorim să identificăm daca o observaţie nouă este normală sau nu. 
Din acest motiv, este foarte important ca setul de antrenare sa nu conţină
decât date normale pentru a putea identifica caracteristicile clasei de referință.

\textbf{Novelty detection} este tipul de detecţie al anomaliilor pe care îl studiem în 
această lucrare.

\section{Detecţia anomaliilor vs clasificare binară}

\subsection{Greşeală comună}

La prima vedere, aceste două probleme par sa coincidă. Un argument ar putea fi că 
detecţia anomaliilor este de fapt o problemă de clasificare binară în care proporţia
celor 2 clase este neechilibrată. Deşi este contraintuitiv, aceasta 
afirmaţie este greşită.

\subsection{Scopul detecţiei anomaliilor}

Diferenţa constă în faptul că detecţia anomaliilor are la bază \textbf{o singură clasa
de referință}. Observaţia fie aparţine acestei clase, fie aparţine oricărei alte clase 
care este diferită de aceasta din urmă. Multitudinea de "clase diferite" ne indică
faptul că este greu să definim ce înseamnă \textit{"diferit de normal"}, dar este relativ uşor
să definim ce înseamnă \textit{"normal"}.
Din acest motiv, 
metoda este cel mai bine utilizată pe seturi de antrenare ce conţin numai clasa de 
referință.

\subsection{Scopul clasificării binare}

La clasificarea binară, cele 2 clase sunt bine 
definite şi de cele mai multe ori, trasăturile lor se suprapun în mai multe locuri. De 
asemenea, această problemă acordă o importanţă \textbf{egală} celor 2 categorii, pe când
eşecul de a semnala o anomalie este în general mult mai dăunător faţă de 
detectarea unei observaţii normale ca fiind anormală. 

